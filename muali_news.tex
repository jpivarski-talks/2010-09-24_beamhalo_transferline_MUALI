\documentclass[compress]{beamer}
\usepackage{ifthen,verbatim}

\newcommand{\isnote}{}
\xdefinecolor{lightyellow}{rgb}{1.,1.,0.25}
\xdefinecolor{darkblue}{rgb}{0.1,0.1,0.7}

%% Uncomment this to get annotations
%% \def\notes{\addtocounter{page}{-1}
%%            \renewcommand{\isnote}{*}
%% 	   \beamertemplateshadingbackground{lightyellow}{white}
%%            \begin{frame}
%%            \frametitle{Notes for the previous page (page \insertpagenumber)}
%%            \itemize}
%% \def\endnotes{\enditemize
%% 	      \end{frame}
%%               \beamertemplateshadingbackground{white}{white}
%%               \renewcommand{\isnote}{}}

%% Uncomment this to not get annotations
\def\notes{\comment}
\def\endnotes{\endcomment}

\setbeamertemplate{navigation symbols}{}
\setbeamertemplate{headline}{\mbox{ } \hfill
\begin{minipage}{5.5 cm}
\vspace{-0.75 cm} \small
\end{minipage} \hfill
\begin{minipage}{4.5 cm}
\vspace{-0.75 cm} \small
\begin{flushright}
\ifthenelse{\equal{\insertpagenumber}{1}}{}{Jim Pivarski \hspace{0.2 cm} \insertpagenumber\isnote/\pageref{numpages}}
\end{flushright}
\end{minipage}\mbox{\hspace{0.2 cm}}\includegraphics[height=1 cm]{../cmslogo} \hspace{0.01 cm} \vspace{-1.05 cm}}

\newcommand{\s}[1]{{\mbox{\scriptsize #1}}}

\begin{document}
\begin{frame}
\vfill
\begin{center}
\textcolor{darkblue}{\Large Muon Alignment News}

\vfill
\begin{columns}
\column{0.75\linewidth}
\begin{center}
\large
Jim Pivarski \hspace{0.5 cm} Gervasio Gomez
\end{center}
\end{columns}

\vfill
24 September, 2010

\end{center}
\end{frame}

%% \begin{notes}
%% \item This is the annotated version of my talk.
%% \item If you want the version that I am presenting, download the one
%% labeled ``slides'' on Indico (or just ignore these yellow pages).
%% \item The annotated version is provided for extra detail and a written
%% record of comments that I intend to make orally.
%% \item Yellow notes refer to the content on the {\it previous} page.
%% \item All other slides are identical for the two versions.
%% \end{notes}

\small

\begin{frame}
\frametitle{Another round of reprocessing!}
\framesubtitle{Here is the timeline (for some value of $N$):}
\begin{itemize}
\item \textcolor{darkblue}{Friday the $N^\s{th}$ muon alignment meeting:} we sign-off:
\begin{itemize}
\item track-based barrel chamber alignment (method only)
\item hardware barrel chamber alignment (final values)
\item whether to use track-based or hardware for this round
\item endcap disk alignment (method only)
\end{itemize}

\item \textcolor{darkblue}{Same day:} tracker alignment becomes frozen (their final deadline)

\item \textcolor{darkblue}{Two days + 1 contingency:} Sasha produces new cross-alignment (GlobalPositionRcd)

\item \textcolor{darkblue}{Next two days + 1 contingency:} track-based barrel chambers are updated and endcap disks are updated (in parallel)

\item Final constants posted on HyperNews \textcolor{darkblue}{whenever they're ready}

\item \textcolor{darkblue}{Friday the $(N+1)^\s{th}$ muon alignment
  meeting:} we sign-off {\it values} of the final constants and pass them to the Muon POG
  for testing

\item \textcolor{darkblue}{Next Monday:} we present the results at the Joint Muon DPG
\end{itemize}
\end{frame}

\begin{frame}
\frametitle{Previous slide's unanswered questions}
\begin{itemize}\setlength{\itemsep}{0.3 cm}
\item \textcolor{darkblue}{What is $N$?}  A compromise between tracker alignment and muon
  validation: to be decided in a global optimization--- we're just
  asking for a (Friday-to-Friday) week for our part

\vspace{0.3 cm}
\textcolor{darkblue}{Answer \#2:} approximately October 15

\item \textcolor{darkblue}{Can we reschedule this meeting?}  Only after this sign-off (last slide)

\item \textcolor{darkblue}{Which alignment updates will be included?}  That depends on $N$,
  but we will improve only what we're comfortable with and fall back
  on what we currently have if they don't make it in time.

  We have a baseline track-based and a baseline hardware
  alignment

\item \textcolor{darkblue}{How will we decide between track-based and hardware in the
  barrel?}  Gervasio and I are proposing a test that should be
  conclusive (see my talk, later today)
\end{itemize}
\end{frame}

\begin{frame}
\frametitle{Alignment pipeline}
\framesubtitle{Improvements to the baseline alignment targeted for this sign-off}
\begin{itemize}\setlength{\itemsep}{0.3 cm}
\item Barrel hardware: simultaneous fit of chambers within supersectors and supersectors

\item Barrel chamber track-based:
\begin{enumerate}
\item understand and control ``variation of residuals within chambers'' effect (nearly done)
\item loosen momentum cut on cosmic rays
\item check alignment with collisions muons
\end{enumerate}

\item Endcap disk track-based:
\begin{enumerate}
\item loosen momentum cut on cosmic rays (demonstrated)
\item align with collisions muons
\end{enumerate}

\item Endcap hardware: improvements to $Z$ positions?
\end{itemize}

\textcolor{darkblue}{\large Anything else?}
\end{frame}

\begin{frame}
\frametitle{Rescheduling this meeting}
\begin{itemize}\setlength{\itemsep}{0.3 cm}
\item Marcus raised a good point: this meeting conflicts the Exotica
  resonances meeting, where $Z' \to \mu\mu$ and $W' \to \mu\nu$ are
  discussed
\begin{center}
\renewcommand{\arraystretch}{1.3}
\begin{tabular}{c p{0.5 cm} c}
Muon Alignment & & Exotica Resonances \\\hline
Friday 15:00--? (CERN-time) & & Friday 14:30--16:30
\end{tabular}
\end{center}

\item At least three of us would like to attend both

\item \textcolor{darkblue}{But this is also an important topic for us:
  these are the two CMS analyses known to benefit from muon alignment}

\item Rescheduling is always tricky, but it would be too disruptive to
  move during this sign-off cycle

\item What do people here think about changing the meeting time,
  effective in late October?
\end{itemize}
\label{numpages}
\end{frame}

%% \begin{frame}
%% \frametitle{Outline}
%% \begin{itemize}\setlength{\itemsep}{0.75 cm}
%% \item 
%% \end{itemize}
%% %% \hspace{-0.83 cm} \textcolor{darkblue}{\Large Outline2}
%% \end{frame}

%% \section*{First section}
%% \begin{frame}
%% \begin{center}
%% \Huge \textcolor{blue}{First section}
%% \end{center}
%% \end{frame}

%% \begin{frame}
%% \end{frame}

\end{document}
